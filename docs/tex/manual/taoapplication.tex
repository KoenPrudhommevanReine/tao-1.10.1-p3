\chapter{TAO Applications using PETSc}
\sindex{application}
\label{chapter:taoapplication}
\label{chapter:petscapp}

The solvers in TAO address applications that have a set of variables, an objective
function, and constraints on the variables.  Many solvers also require derivatives
of the objective and constraint functions.
To use the TAO solvers, the application developer must 
define a set of variables, implement routines that evaluate the 
objective function and constraint functions, and pass this information
to a TAO application object.   

TAO uses vector and matrix objects to pass this information from the
application to the solver.   The set of variables, for instance, is
represented in a vector.
The gradient of an objective function $f: \, \Re^n \to \Re$,
evaluated at a point, is also represented as a vector.
Matrices,  on the other hand,
can be used to represent the Hessian of $f$ or the Jacobian of a constraint
function $c: \, \Re^n \to \Re^m$.  The TAO solvers use
these objects to compute a solution to the application.

The PETSc package provides parallel and serial implementations of these
objects and offers additional tools intended for high-performance 
scientific applications.
The {\bf Vec} and {\bf Mat} types in PETSc
represent the vectors and matrices in a TAO application.
This chapter will describe how to create these an application object
and give it the necessary properties.  This chapter will also describe how to use
the TAO solvers in conjunction with this application object.

\section{Header File}

TAO applications written in C/C++ should have the statement 
\begin{verbatim}
   #include "tao.h"
\end{verbatim}
\noindent
in each file that uses a routine in the TAO libraries.
All of the required lower level include files such as ``tao\_solver.h''
and ``taoapp.h''
are automatically included within this high-level file.

\section{Create and Destroy}
To create an application object, first
declare a {\tt TAO\_APPLICATION} variable.
This variable is only a pointer.  
The application object associated with it can be created using
the routine \findex{TaoApplicationCreate()}
\begin{verbatim}
   TaoApplicationCreate(MPI_Comm, TAO_APPLICATION*);
\end{verbatim}
\noindent
Much like creating PETSc vector and matrix objects, 
the first argument is an MPI {\em communicator}.
An MPI \cite{using-mpi} communicator
indicates a collection of processors that will be used to evaluate the
objective function, compute constraints, and provide derivative information.
When only one processor is being used, the communicator {\tt MPI\_COMM\_SELF}
can be used with no understanding of MPI.
Even parallel users need to be familiar with only the basic concepts 
of message passing and  distributed-memory computing. 
Most applications running TAO in
parallel environments can employ the communicator {\tt
MPI\_COMM\_WORLD} to indicate all processes in a given run.

%The {\tt TAO\_APPLICATION} variable whose address is in the second argument
%will be set the 
The second argument is the address of a  {\tt TAO\_APPLICATION} variable.  This
routine will create a new application object and set the variable, which is a pointer,
to the address of the object.   This application variable can now be used
by the developer to define the application and by the 
TAO solver to solve the application.

Elsewhere in this chapter, the {\tt TAO\_APPLICATION} variable will be referred
to as the {\em application object}\sindex{application object}.

After solving the application, the command \findex{TaoApplicationDestroy()}
\begin{verbatim}
   TaoAppDestroy(TAO_APPLICATION);
\end{verbatim}
\noindent
will destroy the application object and free 
the work space associated with it.

\section{Defining Variables}

In all of the optimization solvers, the application must provide
a {\bf Vec} object of appropriate dimension to represent the variables.
This vector will be cloned by the solvers to create additional work
space within the solver.
If this vector is distributed over multiple processors, it
should have a parallel distribution that allows
for efficient scaling, inner products, and
function evaluations.  This vector can be passed to the
application object using the routine \findex{TaoAppSetInitialSolutionVec()}
\begin{verbatim}
   TaoAppSetInitialSolutionVec(TAO_APPLICATION,Vec);
\end{verbatim}
When using this routine, the application should initialize the vector with
an approximate solution of the optimization problem before calling the
TAO solver.   If you do not know of a solution that that can be used,
the routine
\findex{TaoAppSetDefaultSolutionVec()}
{\tt   TaoAppSetDefaultSolutionVec(TAO\_APPLICATION,Vec);}
can be used to declare variables that will in
be set to zero or some other default solution.  

This vector will be used by the TAO solver to store the solution.
Elsewhere in the application, 
this solution vector can be retieved from the application object 
using the routine \findex{TaoAppGetSolutionVec()}
\begin{verbatim}
   TaoAppGetSolutionVec(TAO_APPLICATION, Vec *);
\end{verbatim}
This routine takes the address of a {\tt Vec} in the second argument and sets it to
the solution vector used in the application.

\section{Application Context}  \sindex{application context}
Writing an application using the {\tt TAO\_APPLICATION} object may require
use of an {\em application context}.
An application context is a structure or object defined by an
application developer, passed
into a routine also written by the application developer, 
and used within the routine to perform its stated task.
 
For example, a routine that evaluates an objective function may need
parameters, work vectors, and other information.   This information,
which may be specific to an application and necessary to evaluate the objective,
can be collected in a single structure and used as one of the
arguments in the routine.
The address of this structure will be cast as type {\tt (void*)} and passed to
the routine in the final argument.
There are many examples of these structures in the TAO distribution.

This technique offers several advantages.
In particular, it allows for a uniform interface between TAO and 
the applications.   The fundamental information needed by TAO 
appears in the arguments of the routine, while data specific to an application
and its implementation is confined to an opaque pointer.
The routines can access information created outside the 
local scope without the use of global variables.
The TAO solvers and application objects will never access this structure, 
so the application developer has complete freedom to define it.  In fact,
these contexts are completely optional -- a NULL pointer can be used.



\section{Objective Function and Gradient Routines}\label{sec:fghj}

TAO solvers that minimize an objective function require
the application to evaluate the objective function.  Some solvers
may also require the application to evaluate
derivatives of the objective function.  
Routines that perform these computations must be identified
to the application object and must follow a strict calling sequence.

Routines that evaluate an objective function $f: \, \Re^n \to \Re$,
should follow the form:
\begin{verbatim}
   EvaluateObjective(TAO_APPLICATION,Vec,double*,void*);
\end{verbatim}
\noindent
The first argument is the application object, the second argument is the
$n$-dimensional vector that identifies where the objective should be evaluated, 
and the fourth argument is an application context.
This routine should use the third argument to return objective value, 
evaluated at the given point
specified the by the vector in the second argument.

This routine, and the application context, should be passed to the 
application object using
the routine \findex{TaoAppSetObjectiveRoutine()}
\begin{verbatim}
   TaoAppSetObjectiveRoutine(TAO_APPLICATION,
                             int(*)(TAO_APPLICATION,Vec,double*,void*),
                             void*);
\end{verbatim}
\noindent
The first argument in this routine is the application object, 
the second argument is a function pointer to the routine that 
evaluates the objective, and the third
argument is the pointer an appropriate application context.  

Although final argument may point to anything, it must be cast as a {\tt (void*)} type.
This pointer will be passed back to the developer in the fourth argument of the
routine that evaluates the objective.  In this routine, the pointer can be cast
back to the appropriate type.  Examples of these structures and there usage
are provides in the distribution.


   Most TAO solvers also require gradient information from the 
application \sindex{gradients}.
  The gradient of the objective function can be specified in a similar manner.
Routines that evaluate the gradient should have the calling sequence
\begin{verbatim}
   EvaluateTheGradient(TAO_APPLICATION,Vec,Vec,void*);
\end{verbatim}
\noindent
In this routine, the first
argument is the application object, the second argument is the variable
vector, the third argument is the gradient, and the fourth argument is
the user-defined application context.  Only the third argument in this
routine is different from the arguments in the routine that evaluates
the objective function.  The numbers in the gradient vector have no
meaning when passed into this routine, but should represent the gradient
of the objective at the specified point at the end of the routine.
This routine, and the user-defined pointer, can be passed to the application
object using the routine: \findex{TaoAppSetGradientRoutine()}
\begin{verbatim}
   TaoAppSetGradientRoutine(TAO_APPLICATION,
                            int (*)(TAO_APPLICATION,Vec,Vec,void*),
                            void *);
\end{verbatim}
\noindent
In this routine, the first argument is the application object, the second argument
is the function pointer, and the third object is the application context, cast
to {\tt (void*)}.

   Instead of evaluating the objective and its gradient in separate
routines, TAO also allows the user to evaluate the function and the gradient
at the same routine.  In fact, some solvers are more efficient when
both function and gradient information can be computed in the same routine.
These routines should follow the form
\begin{verbatim}
   EvaluateFunctionGradient(TAO_APPLICATION,Vec,double*,Vec,void*);
\end{verbatim}
\noindent
where the first
argument is the TAO solver, and the second
argument points to the input vector for use in evaluating the
function and gradient. The third argument should return the
function value, while the fourth argument should return the gradient vector,
and the fifth argument is a pointer to a user-defined context.
This context and the name of the routine should be set with the
call: \findex{TaoAppSetObjectiveAndGradientRoutine()}
\begin{verbatim}
   TaoAppSetObjectiveAndGradientRoutine(TAO_APPLICATION,
                     int (*)(TAO_APPLICATION,Vec,double*,Vec,void*),
                     void *);
\end{verbatim}
\noindent
The arguments of this routine are the TAO application, a
function name, and a pointer to a user-defined context.

The TAO example problems demonstrate the use of these application contexts
as well as specific instances of function, gradient, and Hessian 
evaluation routines.
All of these routines should return the integer $0$ after 
successful completion and a nonzero integer if the function
is undefined at that point or an error occurred.

\section{Hessian Evaluation}
\label{sec:matrixfree}
\label{sec:finitedifference}

Some optimization routines also require a Hessian matrix from the user.
The routine that evaluates the Hessian should have the form:
\begin{verbatim}
   EvaluateTheHessian(TAO_APPLICATION,Vec,Mat*,Mat*,MatStructure*,void*);
\end{verbatim}
\noindent
The first argument of this routine is a TAO application.  The second
argument is the point at which the Hessian should be evaluated.  The
third argument is the Hessian matrix, and the sixth argument is a
user-defined context.
Since the Hessian matrix is usually used in solving
a system of linear equations, a preconditioner for the matrix is often
needed.  The fourth argument is the matrix that will be used
for preconditioning the linear system.  In most cases, this
matrix will be the same as the Hessian matrix.  The fifth
argument is the flag used to set the Hessian matrix and
linear solver in the routine {\tt KSPSetOperators()}.

One can set the Hessian evaluation routine by calling \sindex{Hessian}
\findex{TaoAppSetHessianRoutine()}
\begin{verbatim}
   int TaoAppSetHessianRoutine(TAO_APPLICATION,
                int (*)(TAO_APPLICATION,Vec,Mat*,Mat*,MatStructure*,void*),
                void *)
\end{verbatim}
\noindent
The first argument is the TAO application, the second 
argument is the function that evaluates the Hessian, 
and the third argument is a pointer to a user defined context,
cast as a {\tt void*} pointer.

For solvers that evaluate the Hessian, 
the matrices used to store the Hessian should be set using 
\findex{TaoAppSetHessianMat()}
\begin{verbatim}
   TaoAppSetHessianMat(TAO_APPLICATION,Mat,Mat);
\end{verbatim}
\noindent
The first argument is the TAO application, the second argument is the Hessian
matrix, and the third argument is the preconditioning matrix.  In most applications,
these two matrices will be the same structure.

\subsection{Finite Differences} \sindex{finite differences}
Finite differences approximations can be used to compute the gradient and the
Hessian of an objective
function.  These approximations will slow down the solve considerably and are only 
recommended for checking the accuracy of hand-coded gradients and Hessians.
These routines are 

\findex{TaoAppDefaultComputeGradient()}
\begin{verbatim}
  TaoAppDefaultComputeGradient(TAO_APPLICATION, Vec, Vec, void*);
\end{verbatim}, 


\findex{TaoAppDefaultComputeHessian()}
\begin{verbatim}
   TaoAppDefaultComputeHessian( TAO_APPLICATION, Vec, Mat*, Mat*, 
                                MatStructure*, void*);
\end{verbatim}
\noindent
and \findex{TaoAppDefaultComputeHessianColor()}
\begin{verbatim}
   TaoAppDefaultComputeHessianColor( TAO_APPLICATION, Vec, Mat*, Mat*, 
                                      MatStructure*, void* );
\end{verbatim}
\noindent
These routines can be set using {\tt TaoAppSetGradientRoutine()} and 
{\tt TaoAppSetHessianRoutine()} or through the options database.
If finite differencing is used with coloring, the routine \findex{TaoAppSetColoring()}
\begin{verbatim}
   TaoAppSetColoring(TAO_APPLICATION, ISColoring);
\end{verbatim}
\noindent
should be used to specify the coloring.

It is also possible to use finite difference approximations to directly check
the correctness of an application's gradient and/or Hessian evaluation routines.
This can be done using the special TAO solver {\tt tao\_fd\_test} together with the options
{\tt -tao\_test\_gradient} or {\tt -tao\_test\_hessian}.


\subsection{Matrix-Free methods}
TAO fully supports matrix-free methods. The matrices specified in the
Hessian evaluation routine need not be conventional
matrices; instead, they can point to the data required to implement a
particular matrix-free method.  The matrix-free variant is allowed
{\em only} when the linear systems are solved by an iterative method
in combination with no preconditioning ({\tt PCNONE} or {\tt -pc\_type none}),
a user-provided preconditioner matrix, or a user-provided preconditioner
shell ({\tt PCSHELL}); that is,
obviously matrix-free methods cannot be used if a direct solver is to 
be employed. \sindex{matrix-free options} %\findex{PCSHELL}
Details about using matrix-free methods are provided in the
PETSc  Users Manual.


\section{Bounds on Variables}\label{sec:bounds}

Some optimization problems also impose constraints upon the variables.
The constraints may impose simple bounds upon the variables, or
require that the variables satisfy a set of linear or  nonlinear equations.

The simplest type of constraint upon an optimization problem puts lower
or upper bounds upon the variables. 
Vectors that represent lower and upper bounds for each variable 
can be set with the command  \sindex{bounds}\findex{TaoSetVariableBoundsRoutine}
\begin{verbatim}
   TaoAppSetVariableBoundsRoutine(TAO_APPLICATION, 
                          int (*)(TAO_APPLICATION, Vec,Vec,void*),void *);
\end{verbatim}
\noindent
The first vector and second vectors should contain the lower and upper 
bounds, respectively.
When no upper or lower bound exists for a variable, the bound
may be set to {\tt TAO\_INFINITY} or {\tt TAO\_NINFINITY}.
After the two bound vectors have been set, they may be accessed with the
with the command  {\tt TaoGetApplicationVariableBounds()}.
Since not all solvers use bounds on variables, the user must be careful 
to select a type of solver that acknowledges these bounds.

\section{Complementarity}\label{sec:constraints}

Constraints in the form of nonlinear equations have the form
$C(X) = 0$ where $C: \Re^n \to \Re^m$.
These constraints should be specified in a 
routine, written by the user, that evaluates {\tt C(X)}.
The routine that evaluates the constraint equations should have the form:
\begin{verbatim}
   int EqualityConstraints(TAO_APPLICATION,Vec,Vec,void*);
\end{verbatim}
\noindent
The first argument of this routine is a TAO application object.  The second argument
is the variable vector at which the constraint function should be evaluated.  
The third argument is the vector of function values {\tt C}, and the fourth
argument is a pointer to a user-defined context.
This routine  and the user-defined context 
should be set in the TAO solver with the command
\findex{TaoAppSetConstraintsRoutine()}
\begin{verbatim}
   TaoAppSetConstraintRoutine(TAO_APPLICATION,
                               int (*)(TAO_APPLICATION,Vec,Vec,void*),
                               void*);
\end{verbatim}
\noindent
In this function, first argument is the TAO application,
the second argument is vector in which to store the function values,
and the third argument is a pointer to a user-defined context that will
be passed back to the user.

The Jacobian of the function {\tt C} is the matrix in $\Re^{m \times n}$
such that each column contains the partial derivatives of {\tt f} with respect
to one variable. 
The evaluation of the Jacobian of {\tt f} should be performed in a routine
of the form
\begin{verbatim}
   int J(TAO_APPLICATION,Vec,Mat*,Mat*,MatStructure*,void*);
\end{verbatim}
\noindent
In this function, the second argument is the variable vector at which to 
evaluate the Jacobian matrix, the third argument is the Jacobian matrix,
and the sixth argument is a pointer to a user-defined context.
This routine should be specified using
\findex{TaoAppSetJacobianRoutine()}
\begin{verbatim}
   TaoAppSetJacobianRoutine(TAO_APPLICATION,Mat,
                int (*)(TAO_APPLICATION,Vec,Mat*,Mat*, MatStructure*,void*), 
                void*);
\end{verbatim}
\noindent
The first argument is the TAO application, the second
argument is the matrix in which the information can be stored,
the third argument is the function pointer, and the fourth argument is
an optional user-defined context.
The Jacobian matrix should be created in a way such that the product of 
it and the variable vector can be put in the constraint vector.

For solvers that evaluate the Jacobian, 
the matrices used to store the Jacobian should be set using 
\findex{TaoAppSetJacobianMat()}
\begin{verbatim}
   TaoAppSetJacobianMat(TAO_APPLICATION,Mat,Mat);
\end{verbatim}
\noindent
The first argument is the TAO application, the second argument is the Jacobian
matrix, and the third argument is the preconditioning matrix.  In most applications,
these two matrices will be the same structure.

\section{Monitors}

By default the TAO solvers run silently without displaying information
about the iterations. The user can initiate monitoring with the
command \findex{TaoSetMonitor()} 
\begin{verbatim}
   int TaoSetMonitor(TAO_SOLVER solver,
                     int (*mon)(TAO_SOLVER tao,void* mctx),
                     void *mctx);
\end{verbatim}
\noindent

The routine, {\tt mon} indicates a user-defined monitoring routine
and {\tt mctx} denotes an optional user-defined context for private 
data for the monitor routine.

The routine set by {\tt TaoAppSetMonitor()} is called once during each
iteration of the optimization solver.  Hence, the
user can employ this routine for any application-specific computations
that should be done after the solution update. 
\findex{TaoAppSetMonitor()}.
\begin{verbatim}
   TaoAppSetMonitor(TAO_APPLICATION, 
                    int (*)(TAO_APPLICATION,void*),void *);
\end{verbatim}



% This linear solver object has a method
% that allows the user to adjust convergence tolerances.
% Alternatively, the
% specific data structures of the linear solver can be accessed and modified
% directly.
% comment from Lois: I don't like the wording of this last sentence.
\section{Linear Solvers}\label{sec:TaoLinearSolvers}
One of the most computationally intensive phases of many optimization
algorithms involves the solution of systems of linear equations.  
The performance
of the linear solver may be critical to an efficient computation
of the solution.
Since linear equation solvers often have a wide variety of options 
associated with them, TAO allows the user to access the linear
solver with the command

\findex{TaoAppGetKSP()}
\begin{verbatim}
   TaoAppGetKSP(TAO_APPLICATION, KSP *);
\end{verbatim}

\noindent
With access to the KSP object, users can customize it for their application
to achieve additional performance.


\section{Application Solutions}
Once the application object has the 
objective function, constraints, derivatives, and other features associated with it,
a TAO solver can be applied to the application.  
For further information about how to create a TAO solver, see the previous chapter.

Once the TAO solver and TAO application object have been created and customized, 
they can be matched with one another
using the routine \findex{TaoSetupApplicationSolver()}
\begin{verbatim} 
   TaoSetupApplicationSolver( TAO_APPLICATION, TAO_SOLVER);
\end{verbatim}
\noindent
This routine will set up the TAO solver for the application.  
Different solvers may set up differently, but they typically
create the work vectors and linear solvers needed to find a solution.  
These structures were not created
during the creation of the solver because the size of the application
was not known.
After calling this routine
the routine {\tt  TaoAppGetTaoSolver()} can be used to obtain
the TAO solver object.  If not called directly by the application, 
\texttt{TaoSetupApplicationSolver()} will be executed inside of the
subroutine \texttt{TaoSolveApplication()}.

The routine \findex{TaoGetGradientVec()}
\begin{verbatim}
   TaoGetGradientVec( TAO_SOLVER, Vec*);
\end{verbatim}
\noindent
will set a pointer to a {\tt Vec} to the vector object containing
the gradient vector and the \findex{TaoGetVariableBoundVecs()} routine
\begin{verbatim}
   TaoGetVariableBoundVecs( TAO_SOLVER, Vec*, Vec*);
\end{verbatim}
\noindent
will set the pointers to the lower and upper bounds on the variables  -- if 
they exist.  These vectors may be viewed at before, during, and after
the solver is running.

Options for the application and solver 
can be be set from the command line using the routine \sindex{options} \findex{TaoSetOptions()}
\begin{verbatim}
   TaoSetOptions( TAO_APPLICATION, TAO_SOLVER);
\end{verbatim}
This routine will call  {\tt TaoSetupApplicationSolver()} if it has not been
called already.
This command also provides information about runtime options when the
user includes the {\tt -help } option on the
command line.


Once the application and solver have been set up, the solver can be called using the routine
\findex{TaoSolveApplication()}
\begin{verbatim}
   TaoSolveApplication( TAO_APPLICATION, TAO_SOLVER);
\end{verbatim}
\noindent
This  routine will call the TAO solver.  If the routine  {\tt TaoSetupApplicationSolver()}
has not already been called, this routine will call it.

After a solution has been found, the routine
\begin{verbatim}
   TaoCopyDualsOfVariableBounds( TAO_APPLICATION, Vec, Vec );
\end{verbatim}
\noindent
can compute the dual values of the variables bounds and copy them
into the vectors passed into this routine.

% \section{Extensions}
%int TaoAppSetDestroyRoutine(TAO_APPLICATION taoapp,int (*destroy)(void*),void *mctx)
%int TaoAppAddObject(TAO_APPLICATION taoapp, char *key, void *mctx, int *id)
%int TaoAppQueryForObject(TAO_APPLICATION taoapp, char *key, void **mctx)
%int TaoAppQueryRemoveObject(TAO_APPLICATION taoapp, char *key)
%int TaoAppSetOptionsRoutine(TAO_APPLICATION taoapp,int (*options)(TAO_APPLICATION) )

\section{Linear Algebra Abstractions}

Occasionally TAO users will have to interact directly with the linear
algebra objects used by the solvers.   Solvers within TAO use 
vector, matrix, index set, and linear solver objects that have no
native data structures. Instead they have methods whose implementation
is uses structures and routines provided by PETSc or other external
software packages.

  Given a PETSc {\tt Vec} object {\tt X}, the user can create a {\tt TaoVec} object. \sindex{vectors}
By declaring the variables 
\begin{verbatim}
   TaoVec *xx;
\end{verbatim}
\noindent
the routine
\begin{verbatim}
   TaoWrapPetscVec(Vec,TaoVec **); 
\end{verbatim}
\noindent
takes the {\tt Vec x} and creates and sets {\tt TaoVec *xx} equal to
a new {\tt TaoVec} object.   This object actually has the derived
type {\tt TaoVecPetsc}.
Given a {\tt TaoVec} whose underlying representation is a PETSc {\tt Vec},
the command
\begin{verbatim}
   TaoVecGetPetscVec( TaoVec *, Vec *);
\end{verbatim}
\noindent
will retrieve the underlying vector.
The routine {\tt TaoVecDestroy()} will destroy the {\tt TaoVec} object,
but the {\tt Vec } object must also be destroyed.

The routine \sindex{matrix}
\begin{verbatim}
   TaoWrapPetscMat(Mat,TaoMat **); 
\end{verbatim}
\noindent
takes the {\tt Mat H} and creates and sets {\tt TaoMat *HH} equal to
the new {\tt TaoMat} object.   The second argument specifies whether
the {\tt Mat} object should be destroyed when the {\tt TaoVec} object
is destroy.
This object actually has the derived
type {\tt TaoMatPetsc}.
Given a {\tt TaoMat} whose underlying representation is a PETSc {\tt Vec},
the command
\begin{verbatim}
   TaoMatGetPetscMat( TaoMat *, Mat *);
\end{verbatim}
\noindent
will retrieve the underlying matrix.
The routine {\tt TaoMatDestroy()} will destroy the {\tt TaoMat} object,
but the {\tt Mat } object must also be destroyed.

Similarly, the routine \sindex{linear solver}
\begin{verbatim}
   TaoWrapKSP( KSP, TaoLinearSolver **);
\end{verbatim}
\noindent
takes a {\tt KSP } object and creates a  {\tt TaoLinearSolver}
object.  The 
\begin{verbatim}
   TaoLinearSolverGetKSP( TaoLinearSolver *, KSP *);
\end{verbatim}
\noindent
gets the underlying {\tt KSP} object from the {\tt TaoLinearSolver}
object.


For index sets, the routine
\begin{verbatim}
   TaoWrapPetscIS( IS, int, TaoIndexSet **);
\end{verbatim}
\noindent
creates a {\tt TaoIndexSet} object.  In this routine, however, the
second argument is the local size of the vectors that this object
will describe.  For instance, this object may describe with
elements of a vector are positive.  The second argument should
be be local length of the vector.  The {\tt IS} object will
be destroyed when the {\tt TaoIndexSet} is destroyed.  The routine
\begin{verbatim}
   TaoIndexSetGetPetscIS( TaoIndexSet *, IS *);
\end{verbatim}
\noindent
will return the underlying {\tt IS} object.


\section{Compiling and Linking}

Portable TAO makefiles follow the rules and definitions
of PETSc makefiles.
In Figures \ref{fig:make3} we present a sample makefile.

\begin{figure}[tbh]
{\footnotesize
\begin{verbatim}   
       CFLAGS    = 
       FFLAGS    = 
       CPPFLAGS  =
       FPPFLAGS  =
       
       include ${TAO_DIR}/bmake/tao_common
   
       minsurf1: minsurf1.o tao_chkopts
            -${CLINKER} -o minsurf1 minsurf1.o ${TAO_LIB} ${PETSC_SNES_LIB}
            ${RM} minsurf1.o
\end{verbatim} 
% $
\noindent
}
\caption{Sample TAO makefile for a single C program}
\label{fig:make3}
\end{figure}

This small makefile is suitable for maintaining a single program that
uses the TAO library.  The most important line in this makefile is the
line starting with {\tt include}:

\begin{verbatim}
   include ${TAO_DIR}/bmake/tao_common
\end{verbatim}
\noindent % $
\findex{TAO_LIB}
This line includes other makefiles that provide the needed definitions
and rules for the particular base software installations (specified by
{\tt \$\{TAO\_DIR\}} and {\tt \$\{PETSC\_DIR\}}) and architecture
(specified by {\tt \$\{PETSC\_ARCH\}}), which are typically set as
environmental variables prior to compiling TAO source or programs.  As
listed in the sample makefile, the appropriate {\tt include} file is
automatically completely specified; the user should {\em not} alter
this statement within the makefile.
 
TAO applications using PETSc should be linked with the
to the {\tt PETSC\_SNES\_LIB} library
as well as the {\tt TAO\_LIB} library. This version uses
PETSc 3.1, and the {\tt PETSC\_DIR} variable
should be set accordingly.  Many examples of makefiles
can be found in the {\tt examples} directories.


\section{TAO Applications using PETSc and FORTRAN}
\label{chapter:petscfapp}
%\label{chapter:fortran}

Most of the functionality of TAO can be obtained by people who program
purely in Fortran 77 or Fortran 90.  Note, however, that we recommend
the use of C and/or C++ because these languages contain several
extremely powerful concepts that the Fortran77/90 family does not.
The TAO Fortran interface works with both F77 and F90 compilers.

Since Fortran77 does not provide type checking of routine input/output
parameters, we find that many errors encountered within TAO Fortran
programs result from accidentally using incorrect calling sequences.
Such mistakes are immediately detected during compilation when using
C/C++.  Thus, using a mixture of C/C++ and Fortran often works well
for programmers who wish to employ Fortran for the core numerical
routines within their applications.  In particular, one can
effectively write TAO driver routines in C++, thereby preserving
flexibility within the program, and still use Fortran when desired for
underlying numerical computations.

% \section{Differences between TAO Interfaces for C and Fortran}

Only a few differences exist between the C and Fortran TAO interfaces,
all of which are due to differences in Fortran syntax.  All Fortran
routines have the same names as the corresponding C versions, and
command line options are fully supported. The routine arguments follow
the usual Fortran conventions; the user need not worry about passing
pointers or values.  The calling sequences for the Fortran version are
in most cases identical to the C version, except for the error
checking variable discussed in Section \ref{sec:fortran_errors}.  In
addition, the Fortran routine {\tt TaoInitialize(char *filename,int info)}
differs slightly from its C counterpart; see the manual page for
details.


\subsection{Include Files}
\label{sec:fortran_includes}

%  In Fortran the
%  initialization command has the form
%  \begin{verbatim}
%     call TaoInitialize(character file_name,integer info)
%  \end{verbatim}\noindent
%  \begin{verbatim}
%     call Finalize(integer info)
%  \end{verbatim}\noindent

Currently, TAO users must employ the Fortran file suffix {\tt .F}
rather than {\tt .f}.  This convention enables use of the CPP
preprocessor, which allows the use of the {\em \#include} statements
that define TAO objects and variables. (Familiarity with the CPP
preprocessor is not needed for writing TAO Fortran code; one can
simply begin by copying a TAO Fortran example and its corresponding
makefile.)  

The TAO directory {\tt \$\{TAO\_DIR\}/include/finclude}
contains the Fortran include files
and should be used via statements 
such as the following:
\begin{verbatim}
    #include "include/finclude/includefile.h"
\end{verbatim}
\noindent
Since one must be very careful to include each file no more than once
in a Fortran routine, application programmers must manually include
each file needed for the various TAO (or other supplementary)
components within their program.  This approach differs from the TAO
C++ interface, where the user need only include the highest level
file, for example, {\tt tao.h}, which then automatically
includes all of the required lower level files.  As shown in the
various Fortran example programs in the TAO distribution, in Fortran
one must explicitly list {\em each} of the include files.


\subsection{Error Checking}
\label{sec:fortran_errors}

In the Fortran version, each TAO routine has as its final argument
an integer error variable, in contrast to the C++ convention of
providing the error variable as the routine's return value.  The error
code is set to be nonzero if an error has been detected; otherwise, it
is zero.  For example, the Fortran and C++ variants of {\tt TaoSolveApplication()} are
given, respectively, below, where {\tt info} denotes the error variable:
\begin{verbatim}
   call TaoSolveApplication(TAO_APPLICATION taoapp, TAO_SOLVER tao, int info)
   info = TaoSolveApplication(TAO_APPLICATION taoapp, TAO_SOLVER tao)
\end{verbatim}
\noindent

Fortran programmers can use the error codes in writing their own
tracebacks.  For example, one could use code such as the following:
\begin{verbatim}
   call TaoSolveApplication(taoapp, tao, info)
   if (info .ne. 0) then
       print*, 'Error in routine ...'
       return
   endif
\end{verbatim}
\noindent
In addition, Fortran programmers can check these error codes with the
macro {\tt CHKERRQ()}, which terminates all process when an error
is encountered.  See the PETSc users manual for details.  The most
common reason for crashing PETSc Fortran code is forgetting the final
{\tt info} argument.

Additional interface differences for Fortran users:
\begin{itemize}
\item \texttt{TaoGetConvergenceHistory()} -- returns only the number of 
elements in the history.  Storage for the convergence information must 
be preallocated by the user and then registered with 
\texttt{TaoSetConvergenceHistory()}.
\item \texttt{TaoSetLinesearch()} -- use only the first and fourth 
arguments. The setup, options, view, and destroy routines do not apply.
\end{itemize}

\subsection{Compiling and Linking Fortran Programs}
\label{sec:fortcompile}

Figure \ref{fig:make4} shows a sample makefile that can be used for
TAO Fortran programs.  You can compile a debugging
version of the program {\tt rosenbrock1f} with 
{\tt make rosenbrock1f}.

\begin{figure}[tbh]
{\footnotesize
\begin{verbatim}   
       CFLAGS    = 
       FFLAGS    = 
       CPPFLAGS  =
       FPPFLAGS  =
       
       include ${TAO_DIR}/bmake/tao_common
   
       rosenbrock1f: rosenbrock1f.o  tao_chkopts
                 -${FLINKER} -o rosenbrock1f rosenbrock1f.o ${TAO_FORTRAN_LIB} ${TAO_LIB} \
                                ${PETSC_FORTRAN_LIB} ${PETSC_SNES_LIB}
        ${RM} rosenbrock1f.o
\end{verbatim} % $
\noindent
}
\caption{Sample TAO makefile for a single Fortran program}
\label{fig:make4}
\end{figure}

\noindent
Note that the TAO Fortran interface library, given by {\tt
\$\{TAO\_FORTRAN\_LIB\}}, {\em must} \findex{TAO_FORTRAN_LIB} precede
the base TAO library, given by {\tt \$\{TAO\_LIB\}}, \findex{TAO_LIB}
on the link line.

\subsection{Additional Issues}

The TAO library currently interfaces to the PETSc library for
low-level system functionality as well as linear algebra support.  The
PETSc users manual discusses additional Fortran issues in these areas,
including
\begin{itemize}
\item array arguments (e.g., {\tt VecGetArray()}),
\item calling Fortran Routines from C (and C Routines from Fortran),
\item passing null pointers,
\item duplicating multiple vectors, and
\item matrix and vector indices.
\end{itemize}


\Comment{
\chapter{TAO Applications using ESI Objects}
\label{chapter:esiapp}
Applications using TAO  
can be implemented using one of several techniques.  TAO
currently supports applications developed using PETSc
as well as applications that have been implemented
with the Equation Solver Interface (ESI)~\cite{esi-web-page}.


TAO also supports applications written using the ESI interface. For more
information about this interface, please contact the developers.
}
% \section{TAO Application Object Model}

%%% Local Variables: 
%%% mode: latex
%%% TeX-master: "manual_tex"
%%% End: 
