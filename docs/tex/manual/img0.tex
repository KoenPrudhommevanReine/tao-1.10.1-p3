% part0.tex near line 28
\documentclass[dvipdfm,11pt,openright]{book}\usepackage{}\usepackage{amssymb}\usepackage{moreverb}\usepackage{epsf}\usepackage{float}\usepackage{}\usepackage{afterpage}
\pagestyle{empty}
\thispagestyle{empty}
\vsize=20in\setlength{\textheight}{20in}
\begin{document}
{\catcode`\_=\active\gdef_{{\tt\char`\_}}}
{\catcode`\_=\active\gdef\makeustext{\def_{{\tt\char`\_}}}}
{\catcode`\&=\active\gdef\makeamptext{\catcode`\&=\active\def&{{\tt\char`\&}}}}
\def\eatcode{\catcode`\_=\active\makeustext\makeamptext\eatnext}
\def\eatfile{\catcode`\_=\active\makeustext\makeamptext\eatnextfile}
\def\eatnext#1{{\tt #1}\endgroup}
\def\eatnextfile#1{`{\tt #1}'\endgroup}
\def\file{\begingroup\eatfile}
\def\code{\begingroup\eatcode}
\setcounter{section}{1}
\setcounter{figure}{0}
\setcounter{table}{0}
\setcounter{equation}{-1}
% User definitions begin here
\def\F{\mbox{\boldmath \(F\)}}
\def\R{\mathbb R}
\def\x{\mbox{\boldmath \(x\)}}
\def\findex#1{\index{#1}}
\def\sindex#1{\index{#1}}
\def\Re{\R}
\def\Comment#1{}
\def\rr{\mbox{\boldmath \(r\)}}
\def\half{{\textstyle{\frac{1}{2}}}}
% User definitions end here
% Disable other commands
\def\index#1{\relax}
% Beginning of Environment to be handled
\begin{comment}
The Toolkit for Advanced Optimization (TAO) focuses on the design of large 
scale optimization software, including nonlinear least 
squares, unconstrained minimization, bound constrained 
optimization, and decomposition techniques. 
The solution of such problems 
pervades many areas of computational science and demands robust and 
flexible solution strategies. 
As surveyed by Mor\'e and Wright \cite{optguide93}, 
various software packages are available for solving these 
problems; however, their portability, versatility, and scalability are 
restricted, especially within parallel environments. 
 
The current generation of numerical software generally has a rigid form 
that imposes many limitations, even when restricted to 
uniprocessor architectures. 
In traditional software design, the expressions of 
algorithms make assumptions about the way mathematical objects, such as 
vectors and matrices, are represented by the computer. 
Thus, users are 
forced to convert from the natural representation of data for a particular 
application to one imposed by the software developer, often at the 
expense of considerable overhead.  In addition, library routines are 
often characterized by long and complicated calling sequences, with 
no consistent interface among algorithms that solve a particular class 
of problems. 
 
These issues are magnified by the very nature of multiprocessor 
architectures, since robust and 
efficient implementation of mathematical abstractions involves 
the added considerations 
of parallel data structures and communication. 
An effective software package should 
exploit different parallel programming 
techniques for various phases of the solution process. 
 
Since many application problems require the computational 
power of high-performance computers, a need clearly exists for a 
uniform and flexible framework for developing optimization software and 
solving application programs. 
Our goal is to use object-oriented and component-based 
software engineering techniques to create such an environment. 
 
\end{comment}
% End of Environment to be handled
\end{document}
